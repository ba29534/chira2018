\section{Results}\label{results}

In this section, we present our results on how velocity structure functions (VSFs) reflect the distribution of turbulent power within molecular clouds.
Fig.~\ref{pic:results:vsf_example} shows three examples of VSFs, namely (a) \texttt{M4} $t$~=~1.2~Myr, after self-gravity has been activated in the simulations, (b) \texttt{M3} at $t$~=~3.5~Myr, and (c) \texttt{M3} at $t$~=~4.0~Myr.
Thereby, the solid lines in the examples illustrate the fitted power-law relations as given in Eq.~(\ref{equ:method:fitting}).

\begin{figure*}[!htb]
	\centering
	\includegraphics[width=\textwidth]{vsf_example.pdf}
    \caption{Examples of velocity structure functions as function of the lag scale, $\ell$, and order, $p$. 
    	The dots (connected by dashed lines) illustrate the measured values based on the simulation data. 
        The solid lines represent the power-law relations fitted to the respective structure function.
	}
    \label{pic:results:vsf_example}
\end{figure*}

All plots illustrate the VSFs of the orders $p$~=~1--3 that are computed based on the simulation data.
The examples demonstrate that, in general, the measured VSFs cannot be described by a single power-law relation over the entire range of $\ell$.
Rather they are composed of roughly three different regimes: 
one at small scales with $\ell \lesssim$~3~pc, a second one within 3~pc~$\lesssim \ell \lesssim$~10--15~pc, and the last one at large scales with $\ell >$~15~pc.
Therefore, only the small and intermediate ranges may be described by a common power-law relation.
On larger scales, one observes a local minimum before the VSFs either increase or remain constant.
The location of the minimum, thereby, coincides with the equivalent radius of the cloud, meaning the radius a cloud of given mass would have if it would be a sphere.
Thus, in this context the VSF is an accurate tool to measure the size of a molecular cloud.
On smaller scales, which correspond to individual clumps and cores, one sees significant differences.

Fig.~\ref{pic:results:zeta_all_normal} plots the time evolution of $\zeta$ obtained for all three clouds.
The figure shows several interesting features.
First, initially all calculated values of $\zeta$ are above the predicted values (see Eqs.~(\ref{equ:method:she}) and~(\ref{equ:method:boldyrev})).
This means that the turbulence within the clouds is highly supersonic before the gas begins to react to the activation of self-gravity.
Second, all $\zeta$ decrease with time as the clouds gravitationally collapse.
Therefore, the gas transfers the turbulent power from large to small scales.
This process accelerates the relative motions between cells on all scales and causes a flat or even inverted profile in the VSF.
Third, occasionally one observes bumps and dips in all orders of VSFs (e.g., \texttt{M3} or \texttt{M8} around $t$~=~1.7~Myr). 
These features only last for short periods of time (up to 0.6~Myr), but set in quasi-instantly and represent a complete relocation of the turbulent power on all scales. 

\begin{figure*}[!htb]
\centering
\begin{subfigure}{\textwidth}
    \includegraphics[width=\textwidth]{zeta_normal.pdf}
    \caption{standard analysis: 3D relative velocities, n$_\mathrm{cloud}$~=~100~cm$^{-3}$, $\lambda_J$~=~$4\Delta{}x$.}
    \label{pic:results:zeta_all_normal}
\end{subfigure}

\begin{subfigure}{\textwidth}
    \includegraphics[width=\textwidth]{zeta_1d.pdf}
    \caption{1D analysis: 1D relative velocities measured parallel to the respective axis, n$_\mathrm{cloud}$~=~100~cm$^{-3}$, $\lambda_J$~=~$4\Delta{}x$.}
    \label{pic:results:zeta_all_1d}
\end{subfigure}

\begin{subfigure}{\textwidth}
    \includegraphics[width=\textwidth]{zeta_jeans.pdf}
    \caption{Jeans refinement analysis: 3D relative velocities, n$_\mathrm{cloud}$~=~100~cm$^{-3}$, based on data simulated with the respective Jeans refinement. Note that the analysis is only available for \texttt{M3}.}
    \label{pic:results:zeta_all_jeans}
\end{subfigure}

\begin{subfigure}{\textwidth}
    \includegraphics[width=\textwidth]{zeta_rand.pdf}
    \caption{Density threshold analysis: 3D relative velocities, n$_\mathrm{cloud}$~=~0~cm$^{-3}$, $\lambda_J$~=~$4\Delta{}x$.}
    \label{pic:results:zeta_all_rand}
\end{subfigure}

\begin{subfigure}{\textwidth}
    \includegraphics[width=\textwidth]{zeta_now.pdf}
    \caption{Density weighting analysis: 3D relative velocities, n$_\mathrm{cloud}$~=~100~cm$^{-3}$, but the VSF is calculated without taking density weighting into account, $\lambda_J$~=~$4\Delta{}x$.}
    \label{pic:results:zeta_all_now}
\end{subfigure}

\caption{
	Time evolution of scaling exponent $\zeta$ of the $p^\mathrm{th}$ order VSF. 
	The plots (a), (b), (d), and (e) show the measurements for \texttt{M3} (\textit{left}), \texttt{M4} (\textit{middle}), and \texttt{M8} (\textit{right}).
    Panel (c), though, only illustrate the measurements for \texttt{M3} with the respective Jeans refinement level. 
    The grey dotted vertical lines point to the times than a SN explodes within the vicinity of the corresponding cloud, while the blue areas indicate the time of enhanced mass accretion onto the clouds. 
}
\label{pic:results:zeta_all}
\end{figure*}

Yet, \texttt{M8} seems to develop differently.
At the time the SN, occurring at $t$~=~0.8~Myr, hits the cloud the values of $\zeta$ do not rise, as they have done within the other two clouds, but instead they drop. 
After the shock all scaling exponents grow to levels that are slightly above the pre-shock values, before they slowly decrease again.

Fig.~\ref{pic:results:z_all_normal} shows the time evolution of the scaling of the $p^\mathrm{th}$ order VSF relative to the 3$^\mathrm{rd}$ order VSF scaling, $Z(p) = \zeta(p)/\zeta(3)$, based on the model clouds. 
One sees that most of the time the measured values of $Z(p)$ are in agreement or at least closely approaching the predicted values.

\begin{figure*}[!htb]
\centering
\begin{subfigure}{\textwidth}
    \includegraphics[width=\textwidth]{z_normal.pdf}
    \caption{standard analysis: 3D relative velocities, n$_\mathrm{cloud}$~=~100~cm$^{-3}$, $\lambda_J$~=~$4\Delta{}x$.}
    \label{pic:results:z_all_normal}
\end{subfigure}

\begin{subfigure}{\textwidth}
    \includegraphics[width=\textwidth]{z_1d.pdf}
    \caption{1D analysis: 1D relative velocities measured parallel to the respective axis, n$_\mathrm{cloud}$~=~100~cm$^{-3}$, $\lambda_J$~=~$4\Delta{}x$.}
    \label{pic:results:z_all_1d}
\end{subfigure}

\begin{subfigure}{\textwidth}
    \includegraphics[width=\textwidth]{z_jeans.pdf}
    \caption{Jeans refinement analysis: 3D relative velocities, n$_\mathrm{cloud}$~=~100~cm$^{-3}$, based on data simulated with the respective Jeans refinement. Note that the analysis is only available for \texttt{M3}.}
    \label{pic:results:z_all_jeans}
\end{subfigure}

\begin{subfigure}{\textwidth}
    \includegraphics[width=\textwidth]{z_rand.pdf}
    \caption{Density threshold analysis: 3D relative velocities, n$_\mathrm{cloud}$~=~0~cm$^{-3}$, $\lambda_J$~=~$4\Delta{}x$.}
    \label{pic:results:z_all_rand}
\end{subfigure}

\begin{subfigure}{\textwidth}
    \includegraphics[width=\textwidth]{z_now.pdf}
    \caption{Density weighting analysis: 3D relative velocities, n$_\mathrm{cloud}$~=~100~cm$^{-3}$, but the VSF is calculated without taking density weighting into account, $\lambda_J$~=~$4\Delta{}x$.}
    \label{pic:results:z_all_now}
\end{subfigure}

\caption{
	Like Fig.~\ref{pic:results:zeta_all}, but for the measured self-similarity parameter $Z = \zeta(p) / \zeta(3)$ of the $p^\mathrm{th}$ order VSF. 
	The coloured horizontal lines show the predicted values by \citet{She1994} (dash-dotted lines) and \citet{Boldyrev2002} (dashed lines).
}
\label{pic:results:z_all}
\end{figure*}

The peaks in the $Z$ (for example, in \texttt{M4} at $t$~=~4.1~Myr) occur at the times when the scaling exponents of the VSFs, $\zeta$, reach values close or below 0.
The decrease in $Z$ (for example, in \texttt{M3} around $t$~=~1.8~Myr), on the other hand, occur when SN shocks hit and heavily impact the clouds. 

In the following of this section, we will present how VSFs react to variety of setups that are typically assumed in comparable studies.
We will compare the findings with the results we have obtained with our original setup.
In Sect.~\ref{discussion}, we will discuss and interpret these results in more details.


\subsection{Comparison to Line-of-Sight Velocities}\label{results:1d}

So far, we have seen how the VSF behaves and evolves within the clouds.
By doing so, we derived the relative velocities based on the 3D velocity vectors that we read out from the simulations.
Yet, this method is rather not applicable for many other studies. 
It is, on the one hand, computational expensive or, on the other hand, not possible as observations generally only provide the velocity component along the line-of-sight (los).
Thus, in this subsection we investigate how VSFs derived from 1D relative velocities compare to the 3D VSFs presented before.
Thereby, we define the 1D VSF as follows:
\begin{equation}
	\mathit{S}_p^\mathrm{1D} (\ell) = \frac{\langle \, \rho(\vec{x}) \rho(\vec{x}+\vec{\ell}) \, |\Delta \vec{v} \cdot \vec{e}_i|^p  \, \rangle}{\langle  \, \rho(\vec{x}) \rho(\vec{x}+\vec{\ell}) \, \rangle} ,
    \label{equ:results:def_vsf_1d}
\end{equation}
with $\vec{e}_i$ representing the unit vector along the $i$~=~x-, y-, or z-axis.
Figs.~\ref{pic:results:zeta_all_1d} and~\ref{pic:results:z_all_1d} show measured $\zeta$ and $Z$, respectively, derived based on Eq.~(\ref{equ:results:def_vsf_1d}). 
We see that in most of the cases the 1D and 3D VSFs agree well with each other.
Yet, there are cases in which the 1D VSF evolves temporarily or completely differently than the 3D VSF.
For example, the 1D VSF along the x-axis in \texttt{M3} initially behaves like the corresponding 3D VSF, if though with lower absolute values of $\zeta$ (or higher values of $Z$).
However, within $t$~=~2.5--3.8~Myr the samples diverge. 
While the 3D based $\zeta$ cease further and switch signs, the $\zeta$ based on the 1D VSF along the x-axis show a local maximum before converging with the 3D $\zeta$ again. 


\subsection{The Effect of Jeans Length Refinement}\label{results:refinement}

The results we have discussed so far are based on simulation data as they have been presented in \citetalias{IbanezMejia2016} and \citetalias{IbanezMejia2017}.
Due to the huge computational expense the variety of physical and numerical processes (fluid dynamics, adaptive mesh refinement, supernovae, magnetic fields, radiative heating and cooling, and many more) within those simulations, though, have also demanded some compromises.

One of these compromises has been the Jeans refinement criterion that is part of the AMR mechanisms.
The authors have resolved local Jeans lengths by only four cells ($\lambda_J$~=~$4\Delta{}x$).
This is the minimal requirement for modelling self-gravitating gas in order to avoid artificial fragmentation \citep{Truelove1998}. 
Other studies, for example by \citet{Turk2012}, yet have shown that a significant higher refinement is needed to reliably resolve turbulent structures and flows on scales of individual cells.

In the appendix of \citetalias{IbanezMejia2017}, the authors examine the effect the number of cells used for the Jeans refinement has on the measured kinetic energy.
For this, they have rerun the simulations of \texttt{M3} twice; 
once with a refinement of eight cells per Jeans length ($\lambda_J$~=~$8\Delta{}x$) for the first 3~Myr after self-gravity was activated, and once with 32 cells per Jeans length ($\lambda_J$~=~$32\Delta{}x$) for the first megayear of the cloud's evolution.
The authors show that the $\lambda_J$~=~$32\Delta{}x$ simulations smoothly reveal the energy power spectrum on all scales already after this first megayear.
The other two setups also do this.
However, they need more time to overcome the resonances in the respective power spectra that originate from the previous resolution steps. 
This is why one can only fully reliably trust the findings in this paper after the clouds have evolved for approximately 1.5~Myr \citep[see also][]{IbanezMejia2017,Seifried2017b}.

More importantly, though, \citetalias{IbanezMejia2017} have calculated the difference in the cloud's total kinetic energy as function of time and refinement level.
They found that the $\lambda_J$~=~$4\Delta{}x$ simulations miss a significant amount of kinetic energy, namely up to 13\% compared to $\lambda_J$~=~$8\Delta{}x$ and 33\% compared to $\lambda_J$~=~$32\Delta{}x$.
However, they also observed that these differences peak around $t$~=~0.5~Myr and decrease afterwards again, as the $\lambda_J$~=~$4\Delta{}x$ and $\lambda_J$~=~$8\Delta{}x$ simulations adjust to the new refinement levels.
This, of course, means that the results we have derived from the $\lambda_J$~=~$4\Delta{}x$ simulations always needs to be evaluated with respect to this lack of energy, although the clouds' dynamics is dominated by gravitational collapse.
Yet it also means that the $\lambda_J$~=~$4\Delta{}x$ data become more reliable the longer the simulations have time to evolve.

In this section, we present how the level of Jeans refinement influences the behaviour of the VSFs.
In order to do so, we analyse the data of the $\lambda_J$~=~$8\Delta{}x$ and $\lambda_J$~=~$32\Delta{}x$ simulations in the same way as we have done with the $\lambda_J$~=~$4\Delta{}x$ data: measure the VSFs and analyse the time evolution of $\zeta$ and $Z$.
Figs.~\ref{pic:results:zeta_all_jeans} and \ref{pic:results:z_all_jeans} plot the measure values of $\zeta$ and $Z$ for the $\lambda_J$~=~$8\Delta{}x$ and $\lambda_J$~=~$32\Delta{}x$.
In Fig.~\ref{pic:results:jeans_comp} we directly compare the measurements of all refinement levels relative to $\lambda_J$~=~$4\Delta{}x$.

$\lambda_J$~=~$8\Delta{}x$ shows the same behaviour as $\lambda_J$~=~$4\Delta{}x$ previously, with values in both samples being in good agreement as the top panel of Fig.~\ref{pic:results:jeans_comp} demonstrates. 
Over the entire observed time span, the measured values of $\zeta$ decreases as the VSF become flatter.
At the time the SNe interact with the cloud, the VSFs steeply increase forward larger scales, causing values of $\zeta$ (Fig.~\ref{pic:results:zeta_all_jeans}).
Compared to $\lambda_J$~=~$4\Delta{}x$ sample, the peak in $\zeta$ is smoother and last longer here, reflecting an impact time span of a bit less than 1~Myr.

This is also observable in Fig.~\ref{pic:results:z_all_jeans} where the sink of $Z$ due to the SN shock lasts longer than it has done in within the $\lambda_J$~=~$4\Delta{}x$ simulations. 
Besides this, the time evolution of $Z$ based on the $\lambda_J$~=~$8\Delta{}x$ simulations is as sensitive to the turbulence-related events as it has been for $\lambda_J$~=~$4\Delta{}x$.
The divergence which is produced when gravity has transferred the majority of power to smaller scales occurs at the same time. 
The depth of the sink is thereby a numerical artefact caused by $\zeta(3)$ being equal or close to 0 at this very time step. 

The picture changes when analysing the VSFs based on the $\lambda_J$~=~$32\Delta{}x$ runs (Figs.~\ref{pic:results:zeta_all_jeans},~\ref{pic:results:z_all_jeans}, and~\ref{pic:results:jeans_comp} \textit{bottom} panel).
Here one sees that the measured values of both $\zeta$ (Fig.~\ref{pic:results:zeta_all_jeans}) and $Z$ (Fig.~\ref{pic:results:z_all_jeans}) are similar to those for $\lambda_J$~=~$4\Delta{}x$ for the first 0.2~Myr.
After this short period, though, the evolutions of $\zeta$ diverge. 
While $\zeta(1)$ and $\zeta(2)$ continue to decrease in similar, but lower rates compared to $\lambda_J$~=~$4\Delta{}x$, $\zeta(3)$ increases until it peaks at $t$~=~0.8~Myr, before it falls steeply down again.
At $t$~=~1.2~Myr, the last time step of this sample, the values of all $\zeta$ equal the measurements of $\lambda_J$~=~$4\Delta{}x$ again (see also Fig.~\ref{pic:results:jeans_comp}). 
However, since there is no information of how the $\lambda_J$~=~$32\Delta{}x$ simulations develop further we cannot predict whether this correspondence will continue. 

\begin{figure*}
	\centering
    \includegraphics[width=\textwidth]{comp_jeans.pdf}
    \caption{Caption.}
    \label{pic:results:jeans_comp}
\end{figure*}

The bottom panel of Fig.~\ref{pic:results:jeans_comp} illustrates the different evolutions of measured $\zeta$ and $Z$ in the two samples of simulations more clearly.
One sees that the differences between the simulation samples follow the same pattern for all orders of $p$.
The order of difference, though, increases with the order:
While the values for $\zeta(1)$ are still in good agreement, the measured values of $\zeta(2)$ and $\zeta(3)$ for $\lambda_J$~=~$32\Delta{}x$ are 40\% and 100\% higher than those measured for $\lambda_J$~=~$4\Delta{}x$, respectively.
Consequently, this causes differences in $Z(p)$ within 30--52\% between the simulations.



\subsection{The Effect of Density Thresholds}\label{results:densthres}

Another assumption that significantly influence the structure and evolution of VSFs is the density threshold we have applied to filter out the cells that are part of the clouds.
In this paper, we assume a minimal number density that defines the clouds' volume of $n_\mathrm{cloud}~=$~100~cm$^{-3}$.
This means that, when focusing on the cloud-only matter, we only consider those cells with number densities $n \geq n_\mathrm{cloud}$.
We have chosen this threshold as it roughly corresponds to the density when CO becomes detectable.

However, \citetalias{IbanezMejia2017} show that there is usually no harsh jump in density between the ISM and the clouds. 
Instead, the density increases continuously towards the centres of mass within the clouds. Consequently, introducing a density threshold is a rather artificially, weakly physically motivated distinction between the clouds and the ISM.
Observationally, however, introducing a density (or intensity) threshold is unavoidable, be it due to technical limitations (e.g., sensibility of detector) or the nature of the underlying physical processes (for example, excitation rates, or critical density).
Therefore, it is important to study to which extend a density threshold influences the VSF and its evolution.

In Figs.~\ref{pic:results:zeta_all_rand} and~\ref{pic:results:z_all_rand} we show our measurements for $\zeta$ and $Z$, respectively, without introducing an density threshold.
This means that we now do not only measure the relative velocities between cells that are part of the clouds' volume, but between cells within the entire 400$^3$ cell cut-outs that contain both the clouds and the ISM.
For our calculations we apply the same methods as described in Sect.~\ref{methods}. 
Yet, in order to reduce the computational effort we have chosen only a randomly positioned subset of cells within the entire cubes. 
This means that the results shown here are based on 3,200,000 cells (=~5\% of the entire 400$^3$ cube).
As the random selection of cells and the clouds make up only a small fraction of the volume within the cubes, this approach makes it more like to pick locations within the modelled ISM, rather than cells within the clouds. 
Please note, that by doing so we only influence the choice of $\vec{x}$ in Eq.~(\ref{equ:method:def_vsf}), not $\vec{x+\ell}$ as we still compute the relative velocities relative to every other cell at 3D lag distance $\ell$ within the entire cube. 

The figures clearly illustrate that the measurements in the samples without density threshold completely differ from those with the density threshold.
Fig.~\ref{pic:results:zeta_all_rand} shows that the measured values of $\zeta$ are by far higher in the ISM than in the cloud-only sample.
Furthermore, although we see a similar decline of $\zeta$ in \texttt{M4} and \texttt{M8} as the gas contracts under the influence of gravity in the vicinity of the clouds, $\zeta$ generally evolve differently here than what we have observed for the cloud-only matter.
E.g., the pronounced features that have reflected the interactions between the clouds' mass and shock waves are either not as significantly visible or not present at all.
We see a high rate of random fluctuations in the evolution of $\zeta$, as well.
That those fluctuations really represent the turbulent nature of the ISM gas and not a super-position of multiple shocks, that are too weak to influence the compact clouds, becomes evident in the evolution of $Z$ (see Fig.~\ref{pic:results:z_all_rand}). 
Contrary to all of our other test scenarios, all $Z$ here are constant in time and within all clouds, with values slightly lower than those predicted by \citet{She1994} for filamentary flows.


\subsection{The Effect of Density Weighting}\label{results:densweight}

As mentioned previously, Eq.~(\ref{equ:method:def_vsf}) represents the definition of the density-weighted VSF.
The density weighting is required for this study to account for obtaining smooth density distributions on Eulerian grids from the simulations, instead of using individual Lagrangian test particles (or eddies) as one is actually supposed to.
Thus, a natural question to ask is how the two ansatzes compare with each other.
There are a few studies that have targeted this question \rc{(ref!!!)}. 
Yet all of them considered turbulent flows in sterile, homogeneous environments that are not comparable to our clouds.
Other studies, like \citet{Padoan2016a}, use both methods, but not on the same set of data. 

In this section, we investigate the influence of density weighting on VSFs.
For this, we repeat the analysis, but now using the non-weighted VSF as given by,
\begin{equation}
	\mathit{S}_p (\ell) = \langle \, |\Delta \vec{v}|^p  \, \rangle = \langle \, |\vec{v}(\vec{x}) - \vec{v}(\vec{x} + \vec{\ell})|^p  \, \rangle .
    \label{equ:results:def_vsf_no}
\end{equation}

Figs.~\ref{pic:results:zeta_all_now} and~\ref{pic:results:z_all_now} show the measured values of $\zeta$ and $Z$ derived from the non-weighted VSFs based on the $\lambda~=~4\Delta x$, respectively.

Comparing the weighted and non-weighted samples, we see the following:
The non-weighted $\zeta$ (Fig.~\ref{pic:results:zeta_all_now}) and $Z$ (Fig.~\ref{pic:results:z_all_now}) trace the interactions between the gas of the clouds and the SN shocks in the same way as we have seen it for the density-weighted VSF.
In \texttt{M3} and \texttt{M8} we also see that the values of $\zeta$ decrease as the clouds evolve. 
The measurements in \texttt{M4}, however, are almost constant over time. 
In all the cases, the values of $\zeta$ never cease below 0.5; a behaviour that clearly differs from what we have observed in the density-weighted VSFs.

For the measured $Z$ values we see a similar behaviour. 
As before, they are almost constant over time and in good agreement with predicted values.
The only derivations we see compared to the previous results are caused by the interactions between the clouds and incoming SN shock fronts, with levels of disturbance that are identical with our measurements using density-weighted VSFs. 
Yet, the evolution of $Z$s is smoother here as we do not measure any sign inversion of $\zeta$.
This means that we do not have any strong peaks in $Z$ indicating the time when most of the turbulent power is transferred to the small scales. 

Fig.~\ref{pic:results:comp_weighting} summarises the comparison of $\zeta$ (\textit{top}) and $Z$ (\textit{bottom}) measured with the density-weighted (\textit{abscissas}) and non-weighted VSFs (\textit{ordinates}) for all refinement levels.
The figure clearly shows that the measurements only agree well for the highest refinement level with $\lambda~=~32\Delta x$.
Yet we would need more data point to be sure that this correlation is indeed real.
With lower refinement level the measurements correlate less well with each other. 
However, the differences in the samples appear dominantly when the density-weighted $\zeta$ cease below $\approx$0.5, which is the global minimum for the non-weighted $\zeta$. 
This means that none of the $\zeta$ computed in all clouds and refinement levels with the non-weighted VSF is measured to be below 0.5.
Thus, we can trace the discrepancies between the scaling exponents of density-weighted and non-weighted VSFs back to the disability of the non-weighted VSFs to follow the transition form large-scale driven to small-scale dominated distributions of turbulent power. 

\begin{figure*}
	\centering
    \includegraphics[width=\textwidth]{comp_weighting.pdf}
    \caption{ Comparison of $\zeta$ (\textit{top}) and $Z$ (\textit{bottom}) measured based on density-weighted VSFs (\textit{abscissas}) and non-weighted VSFs (\textit{ordinates}). }
    \label{pic:results:comp_weighting}
\end{figure*}
