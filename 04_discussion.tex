\section{Discussion}\label{discussion}

The examples in Fig.~\ref{pic:results:vsf_example} illustrate how the clouds and their VSFs react to different scenarios that affect the turbulent structure of the entire clouds. 
In Fig.~\ref{pic:results:vsf_example}a one sees the standard case where turbulence is fully developed, driven on large scales and decays to small scales.
This is the dominant case within the first $\sim$1.5~Myr of the simulations.
During this interval of time the clouds experience the effect of self-gravity for the first time in their evolution and need to adjust to this new condition.
Until this is the case, their VSFs are dominated by the freely cascading turbulence that previously dominated the kinetic structure of the clouds.
Furthermore, this implies that we can only reliably examine turbulence within the simulations after 1.5~Myr and carefully need to take this into account in the further discussion \citep[see][]{IbanezMejia2017,Seifried2017b}.

The other examples represent the clouds at later stages of their evolution when the VSFs are dominated by sources that drive the turbulence within the clouds in a more extreme way.
Fig.~\ref{pic:results:vsf_example}b shows the VSF of \texttt{M3} at a time when the cloud has just been hit by a supernova (SN) shock front. 
One clearly sees how the amplitude of the VSFs is increased by one to two orders of magnitudes compared to the previous example.
Especially the power of small scale turbulence ($\ell \lesssim$ few parsecs) is highly amplified as result of the shock, while it reduces the equivalent radius of the cloud.
Despite the increase of turbulent power at small scales, there is a great amount of energy injected at large scales, as well.
All this results in a steeper scaling of the VSF.
However, the effect of SN shocks last for only a short period of time (see below).

The last example, Fig.~\ref{pic:results:vsf_example}c, demonstrates the imprint of gravitational contraction.
Here, the VSF is almost flat, or even slightly increasing towards smaller separation scales. 
This kind of profile is typical for gas that is self-gravitationally contracting \citep{Boneberg2015,Burkhart2015} since gas moves into the inner regions of the cloud, reducing the average lag distances, but not necessarily the relative velocities.
The latter may even be accelerated by the infall.
As a consequence, large amounts of kinetic energy are transferred to smaller scales which flattens the corresponding VSF.

These interpretations of the examples are verified by the entire time evolution of the clouds. 
Fig.~\ref{pic:results:zeta_all_normal} shows that the measured $\zeta$ cease with time as the clouds contract under the influence of self-gravity.
This is evident as the density and boundness of the clouds increases with as, as well as further gas falls into them \citepalias{IbanezMejia2017}.

However, one also sees peaks in the profiles of $\zeta$ that represent a temporary deviation from the global evolution of the clouds.
The origin of these features can easily be traced back to the SNe occurring in the environment of the clouds.
To visualise this better, we add marks to Fig.~\ref{pic:results:zeta_all} that represent the times when the SNe explode and the periods of time when the clouds are heavily accreting gas.
The data are taken from \citetalias{IbanezMejia2017}, where mean distances between the sites of SNe and the centres of the clouds can also be found.
Considering that the SNe shock fronts move at speeds of 50--100~km~s$^{-1}$ through the ISM and with distances between 30--100~pc to the clouds, the shocks need around 1~Myr, on average, to reach the clouds.
Thus, one cannot only relate the major gas accretion events of the clouds to the arrival of those SNe that indeed affect the evolution of the clouds, but also all significant variations in the evolution of $\zeta$.
In most of the cases, the SNe inject turbulent power into the systems which amplifies the VSFs at all scales, but in particular the large scales.
The latter causes an elevation of $\zeta$.
Yet, the interaction between the clouds and the shocks lasts only for less than 0.6~Myr, after which $\zeta$ evolves back into the pre-shock conditions

The behaviour of \texttt{M8} appears to contradict the explanations we have given for the evolution of turbulence within \texttt{M3} and \texttt{M4}.
However, looking at \texttt{M8} in detail this is not the case.
\citetalias{IbanezMejia2017} show in their Fig.~5 that \texttt{M8} does not react as strongly to the onset of self-gravity as the other clouds do.
For example, the mass of \texttt{M8} remains almost constant for the first megayear while the other clouds strongly accrete gas from the ISM.
Similarly, the velocity dispersion within \texttt{M3} and \texttt{M4} increases within the first 1.5~Myr while it is close to constant in \texttt{M8} until the SN shock hits the cloud.
Thus, we explain the evolution of $\zeta$ as follows:
In the beginning, \texttt{M8} evolves only slightly.
It loses a small fraction of gas \citepalias[Fig.~5, \textit{middle} panel]{IbanezMejia2017}.
Therefore, the VSFs increase towards larger separations.
At $t$~=~1.5~Myr the shock front of the $t$~=~0.8~Myr SN hits the cloud.
That causes a short periods of extreme gravitational collapse, traced by the dip in $\zeta$.
As in the other clouds, the gas relaxes rapidly after the shock.
However, the cloud continues to gravitationally contract in a more regular way.
That is seen as slow decrease of $\zeta$ from $t$~=~1.8~Myr on. 

In summary, one can say that the scaling exponent, $\zeta$, that is obtained by fitting a power-law relation onto the measured velocity structure function, is a useful tool to understand and evaluate the time evolution of turbulence within molecular clouds. 
It is not only sensitive to both external (SNe) and internal (gravitational collapse) driving sources, but also reacts differently to the individual driving sources. 

However, this diagnostic requires a series of time steps to be significant.
According to \citet{She1994} and \citet{Boldyrev2002}, $\zeta$(3) is supposed to be equal or larger than unity whenever the gas experiences supersonic turbulence.
Although this is definitely the case in the model clouds \citepalias{IbanezMejia2016,IbanezMejia2017}, one sees that $\zeta$(3) declines far below 1 due to gravitational collapse.
Thus, measuring $\zeta$ for individual moments in time, as observations would do, cannot fully describe the turbulence of molecular clouds.

The principle of "extended self-similarity" \citep[Sect.~\ref{methods:vsf}]{Benzi1993} offers a solution to this problem.
The principle reflects the nature and the properties of intercloud turbulence, even when it is dominated by gravitational collapse.
However, we also detect strong deviations that either reduce or increase the measured values of $Z$.
Those derivations can be related to the physical forces that currently dominate the clouds.

We observe a similar behaviour in the time evolution of the self-similarity parameter $Z(p) = \zeta(p) / \zeta(3)$, shown in Fig.~\ref{pic:results:z_all_normal}.
The peaks in the $Z$ (for example, in \texttt{M4} at $t$~=~4.1~Myr) occur at the times when the scaling exponents of the VSFs, $\zeta$, reach values close or below 0.
That means that the VSF becomes flat ($\zeta$~=~0) or increases forward smaller scales ($\zeta~<~0$). 
In this case, self-gravitational contraction clearly dominates the cloud's evolution.
It transfers the majority of turbulent power from the large to the small scales, where filaments and fragments are forming and accreting gas.
The decrease in $Z$ (for example, in \texttt{M3} around $t$~=~1.8~Myr), on the other hand, occur when SN shocks hit and heavily impact the clouds. 
This causes a sudden, but heavy increase of turbulent power on all scales, though on the larger scales more than on the smaller, resulting in a steepening of the VSF towards larger scales and an increase in $\zeta$.
The VSFs become more sensitive to the shock the higher their order is.
Therefore, the $\zeta(3)$ increases more rapidly than $\zeta(2)$ and $\zeta(1)$, causing $Z(2)$ and $Z(1)$ to decrease.

In summary, $Z$ is not as sensitive to gravitational contraction as $\zeta$, although it is very sensitive to the inversion of scales that are dominant in turbulent power.
It also traces the impact of SN shocks on the clouds.
Whenever neither of these two extreme scenarios is acting on the clouds, the values of $Z$ are close to the predicted values (see discussion below).
This means that $Z$ is, as $\zeta$, a good tracer for the dominant forces driving the turbulence within the cloud, both internally and externally, as long as they are heavily impacting the cloud.
The disadvantage is that one cannot use $Z$ to distinguish between freely-floating fully-developed and moderate gravitational contraction, as one can do it with $\zeta$.
However, the advantage is that a single-epoch observation of $Z$ is sufficient to trace extreme motions and driving sources within observed molecular clouds.

At the times when the clouds are not impacted by the above described extreme cases, one sees that the ratio of the VSF scaling exponents is mostly in agreement with predicted values for self-similar turbulence, although the clouds are dominated by gravitationally contracting motions during their evolution.
The measured $Z$, thereby, do not uniquely follow the predictions of only \citet{Boldyrev2002} or \citet{She1994}, but are normally between the predicted values of both theories. 
Recalling that both studies describe supersonic, fully developed turbulent flows, the only difference between them are the different geometries along which they allow the gas to flow.
In the case of \citet{Boldyrev2002}, the gas flows are sheet-like, while they are filamentary in the work by \citet{She1994}.

Since \texttt{M3} is heavily impacted by many SN shocks and heavy collapse motions, it does not allow us to make strong predictions about its 'steady-state' evolution.
The other two clouds, on the contrary, are less externally impacted.
This allows us to relate the developments of measured $Z$ to the internal evolution of the clouds.
In \texttt{M8}, $Z$ follows the predictions by \citet{She1994} for most of the cloud's evolution. 
This shows that \texttt{M8} mostly transfers gas along filamentary substructures which means that the cloud is highly hierarchically structured already at the beginning of the simulations and before self-gravity is active.
This also implies that the formation of filamentary structures does not require gravity \citep[e.g.,][]{Federrath2016}.
The fact that we do not detect fragments before the clouds have evolved under the influence of self-gravity for at least one megayear \citepalias[see][]{Chira2017}, however, demonstrates that gravity is essential for the fragmentation of filaments and formation of further substructures.

The evolution of $Z$ in \texttt{M4} shows a slightly different picture that still contains significant conclusion.
While the values of $Z$ in \texttt{M8} are mostly constant over time, they decrease in \texttt{M4}, starting at values that are in agreement with the sheet-like flows of \citet{Boldyrev2002} to those that are predicted for filamentary flows by \citet{She1994}. 
This demonstrates that \texttt{M4} develops its hierarchical structure as it evolves under the influence of self-gravity. 
As a consequence, its turbulent structure becomes more dominated by the filaments with time which causes the decline of $Z(p)$.
Considering that the gas of \texttt{M4} is indeed first flattened into more sheet-like geometry through the impact of the SNe \citep{IbanezMejia2017} this observation is very interesting.
It agrees with the studies, for example by \citet{Lin1965} or \citet{McKee2007}, that claim that molecular clouds are supposed to collapse in a dimension-losing, outside-in fashion that does not require any initial substructures within the clouds.

In summary, $Z$ reflects the global geometry of turbulence within molecular clouds. 
Thereby, the measure values of $Z$ are in good agreement with predictions from self-similarity theory, as long as one carefully ensures that the dominating turbulence mode in the cloud matches the modes that are considered in the respective theory.
This makes $Z$ a reliable parameter for examining turbulence modes in observational studies.
Furthermore, the time evolution of $Z$ shows how the geometry of turbulence is changed due to the gravitational contraction, e.g.~from sheet-like to filamentary vortices, accompanies the change of basic parameters describing the turbulent structure of the entire cloud. 
$Z$ can also deviate from the predicted values. 
This, however, only occurs in time spans of extreme turbulence driving, such as SN shocks.
Fortunately, these extreme driving sources affect $Z$ differently:
SN shocks decrease $Z$, while gravity causes a quasi-momentary peak in $Z$.
In both cases, $Z$ relaxes to the pre-perturbation values within a short time.

\subsection{Comparison to Line-of-Sight Velocities}\label{discussion:1d}

In this paper, we do not only aim to analyse the mechanisms that drive turbulence within simulated molecular clouds, but also to provide a practical tool that can be used understand the dynamical processes within the observed ISM better.
For the latter, it is necessary to examine its applicability under and the dependencies of the obtained results on all relevant conditions. 
As we have chosen to use VSFs for our analysis it is evident that we first need to investigate how the function reacts to the number of measured velocity vector components. 

We recall that Equ.~(\ref{equ:method:def_vsf}) gives the VSF  as the mass-weighted average of relative velocities between the clouds' gas cells.
This means that for a proper one needs all three components of the three-dimensional (3D) velocity vector for each contributing cell.
Observations, however, normally measure only the one-dimensional (1D) component along the line-of-sight (los), also known as local standard of rest velocity.
If the gas moves exclusively along the los both the 3D and 1D VSF will return the same results. 
If the gas is, contrary, driven into directions perpendicular to the los only the 1D VSF will be absolutely 0.
 
In this section, we discuss the results considering this aspect.
Hence, we use the same data of the model clouds as before and produce three subsamples by projecting the 3D velocity vectors onto the three  major axes x, y, and z, respectively.
The derived $\zeta$ and $Z$ for each subsample are shown in Figs.~\ref{pic:results:zeta_all_1d} and~\ref{pic:results:z_all_1d}.

In Sect.~\ref{results:1d} we have seen that the $\zeta$ and $Z$ derived from the 1D VSFs generally evolve similarly as those derived from the 3D VSFs.
Yet, we have also seen that individual sight lines may evolve differently.
Those differences are generated when the gas is significantly more driven into the corresponding sight lines than into the others (analogously to the simple scenario described above). 

For example, for the first two megayears of the evolution of \texttt{M4} the values of $Z$ along the y axis are significantly higher than those observed along the other axes and the predicted values.
Recalling that a higher value of $Z$ correspond to an episode of very strong gravitational contraction, we can conclude that within this time span \texttt{M4} is dominantly collapsing along its y axis. 
The values of $Z$ that are measured for the other two axes agree with this conclusion as they are best predicted by \citet{Boldyrev2002} who describes a sheet-like turbulence. 
Note that this effect is only visible as we analyse the three dimensions separately, as the 3D VSFs (see Fig.~\ref{pic:results:z_all}) do not mirror the driving of the gas along the y axis at all.

In summary, we see that for a fully developed 3D turbulent field we expect that 1D VSFs behave similarly to 3D VSFs.
However, when there is preferred direction along which the gas flows the 1D and 3D VSFs differ significantly from each other. 
Thus, it is predictable that observed VSFs can reflect the nature of turbulence within molecular clouds unless there is clear evidence that the gas is driven into a particular direction (e.g., by an HII region, or SN shock front).


\subsection{The Effect of Jeans Length Refinement}\label{discussion:refinement}

In Sect.~\ref{results:refinement}, we motive the need to investigate the influence of Jeans refinement on VSFs.
In particular, we examine how the choice of using the minimally required refinement compares to a generally recommended refinement level, including the differences in total energy and resonances seen in the power spectra. 
To do this, we use the data by \citetalias{IbanezMejia2017} that resolve the Jeans length in \texttt{M3} twice ($\lambda_J$~=~$8\Delta{}x$) and eight times ($\lambda_J$~=~$32\Delta{}x$) finer as the original simulations ($\lambda_J$~=~$4\Delta{}x$).

In Fig.~\ref{pic:results:jeans_comp} we have seen that the choice of refinement level must not have a significant influence on the measurements and evolution of both $\zeta$ and $Z$. 
$\lambda_J$~=~$4\Delta{}x$ and $\lambda_J$~=~$8\Delta{}x$ are in very good agreement with each other.
This means that, although refining Jeans lengths with 4~cells misses about 13\% of kinetic energy, the effect on the structure and behaviour of the turbulence is rather small and not traced by a VSF analysis.

However, the bottom row of Fig.~\ref{pic:results:jeans_comp} also shows that such an agreement is not self-evident as $\lambda_J$~=~$32\Delta{}x$ differs more from $\lambda_J$~=~$4\Delta{}x$ the higher the order of the VSF is.
Following the explanations in Sect.~\ref{results}, the behaviour of $\zeta$ and $Z$ in the $\lambda_J$~=~$32\Delta{}x$ runs corresponds to the reaction of the cloud's gas to a shock wave running through the cloud; caused by a supernova that exploded before $t$~=~0~Myr. 
Indeed one sees a SN at a distance of 172~pc at $t$~=~-1.11~Myr. 
Due to the distance the shock front is too weak to effectively compress the gas within \texttt{M3}.
This is why it has not been detected previously in the less refined samples.
However, the SN explodes far below the mid-plane of the modelled disk galaxy, in a region without dense gas.
This means that the shock wave that has been injected by the explosion is less damned as it propagates through the ISM. 
By the time the front arrives at \texttt{M3} it is still energetic enough to drive strong winds, with velocity above 300~km~s$^{-1}$, at the closer edge of the cloud. 
This causes an increase of VSFs at longer lag scales and the increase of $\zeta$, as well as the drop in $Z$.

Thus, derivations of measured $Z$ to predicted values do only trace external turbulent driving.
Whether the source is a SN shock front propagating through the cloud or a strong wind, cannot be distinguished by $Z$ only. 
Yet, $Z$ remains a fine probe for the geometry of turbulence and the scales at which turbulence is driven.


\subsection{The Effect of Density Thresholds}\label{discussion:densthres}

In Sect.~\ref{results:densthres} we test the influence of the density threshold that defines the volume of our interest on the evolution of VSFs.
In this particular case, we normally define our clouds as volumes of connected gas cells with densities above a certain value n$_\mathrm{cloud}$~=~100~cm$^{-3}$ \citetalias[see][]{IbanezMejia2016,IbanezMejia2017}. 
Although this approach is in agreement with typical post-processing methods of observational data, by doing so we only work with $\leq$1.5\% of the cubes' volumes.

As the post-processing of the entire data cubes, which would be result when removing the density threshold and setting n$_\mathrm{cloud}$~=~0~cm$^{-3}$, would be too computational expensive, we have decided to reduce the number of considered radomly chosen cells to 5\% of the total number of cells.
We emphasise again that this does not mean that we only calculate the relative velocities between those 5\% of cells.
Rather this subsample of cells represent the starting vectors $\vec{x}$ to which the velocities of all other cells  $\vec{x} + \vec{\ell}$ in the same cube are compared to.
This way we reduce the risk of ignoring or emphasising any spatial direction or angle.

Further note that by choosing the starting points randomly we ensure that all parts of the cubes are considered. 
As a consequence, there is only a verys little likelihood (5\%~$\times$~1.5\%~=~0.075\%) that cells that are associated with the clouds are chosen.
Therefore, we emphasise that it is very likely that the two subsamples (no density threshold and cloud-only) do not necessarily have a common set of starting vectors.
Yet, as we here still calculate relative velocities to all other cells in the cube, and that includes all cells making up the clouds, the resulting VSFs of the n$_\mathrm{cloud}$~=~0~cm$^{-3}$ sample also consists of most of the information of the inner-cloud interactions.

In Sect.~\ref{results:densthres} we see that the structure and evolution of VSFs strongly depends on whether or not there is a density threshold. 
In the previous case, where n$_\mathrm{cloud}$~=~100~cm$^{-3}$, we have seen a mostly straight decline of $\zeta$ and a rather, but not completely constant evolution of $Z$ over time that reflect the conteraction of the clouds due to self-gravity.
Here, in the n$_\mathrm{cloud}$~=~0~cm$^{-3}$ sample, however, we see a completly different picture.
There is still a slightly declining trend in $\zeta$.
Yet, the evolutions are dominated by random fluctuations.

In case of $Z$, however, we see that the constant trend, that we have seen in the previous scenarios, does not only continue, but becomes completely constant in time and space. 
This means that the source that drives the turbulence on the scales of the modelled ISM is constantly and isotropically acting on the diffuse gas.
Thus, the ISM matter around the clouds is constantly highly driven by different sources, like SN shocks or galactic shears.
Consequently, individual events, like a single SN, do not have as a high impact on the state of turbulence of the ISM as on the gas within the clouds which is comparably quite.

We conclude that the decision whether or not a density threshold is used for ascertaining insights on the turbulent composition of the observed gas has a significant and direct influence on the resulting VSFs.
As mentioned previously, applying a density threshold is mosten unavoidable as it is a straight-forward approach to filter the data on the actual area of interest.
In observational studies it is even always present as minimal collision rates for excitation or the sensitivity of detectors automatically result in implicite intensity and density thresholds. 
Although we have only tested two specific setups in this context we have seen the significance of a proper choice of the density threshold, as well as a proper discussion of the obtained results considering the used threshold as one of the defining parameters.


\subsection{The Effect of Density Weighting}\label{discussion:densweight}

In this subsection, we discuss the effect the ambiguous definition of VSFs has on the measurements. 
With this we mean that the VSF can be computed with or without taking density weighting into account.
This ambiguity originates in the type of data one analyses.
In general, the VSF considers the average relative velocity between each two Lagrangian particles.
As each fluid particle stores its complete set of information its mass, and therefore its inertia, is automatically considered when iterating its velocity; and Eq.~(\ref{equ:results:def_vsf_no}) is sufficient to derive the VSF of the fluid properly.
In our case, however, we examine Eulerian data that store the parameter of the fluid on a static grid. 
The information of individual particles and vortices are thereby averaged over the grid.
Consequently, it is important to consider this when computing the VSF; which is done by using the density-weighted definition of the VSF given in Eq.~(\ref{equ:method:def_vsf}).

Although the motivation why the different definitions exist and are used for non-overlapping samples of data, it is not yet clear how the results obtained with the two approaches relate to each other. 
This we do by applying both ansatzes on our data.
The results are presented in Sect.~\ref{results:densweight}.
There we see considering the density weighting or not does not have a significant effect on the data, but only as long as the turbulence is dominated by the large scales. 
However, as the clouds evolve the differences increase because the non-weighted VSFs never drop below 0.5.
This is because the non-weighted VSF treats all cells equally, no matter whether they represent a volume element of the clouds or one of the ISM, while the weighted VSF gives more weight to the matter within the clouds.
As long as the turbulence is dominated by large scales this does not cause a notable difference as these long separations, considering the density threshold, represent cells on the outer surface of the clouds, with similar densities and conditions.
The small lag scales, contrary, reflect how the conditions change within the clouds, with density gradients becoming larger towards the centre of the clouds and increasing faster than the relative velocities of neighbouring cells.
Here, the density weighting becomes crucial to compensate for the fact that the higher density represents a higher number of particles in a Lagrangian framework.
However, if one does not consider this (as we have done in Sect.~\ref{results:densweight}) the inner regions of the clouds are not processed correctly.
This ends in a situation where the large scales never seem to loose their dominance, and $\zeta$ never becomes close or less than 0; a situation that does not reflect the reality.

Nevertheless, Fig.~\ref{pic:results:z_all_now} illustrates that these differences can only be traced by $\zeta$ alone. 
Besides the features created when $\zeta$ becomes close to 0 in the density-weighted VSFs, the measured $Z$ have similar values, evolve and reacts to external driving mechanisms similarly independently based on which approach they are computed. 
This observation is true for all Jeans refinement levels, as Fig.~\ref{pic:results:comp_weighting} demonstrates.

We conclude that deriving the VSF from smooth density distributions without considering density weighting does not affect the behaviour of $\zeta$ and $Z$, as long as the turbulence is dominated by large scale vortices, while it has a significant effect on the measurements when the small scales become dominant.
The latter is particularly important as this finding directly influences the conclusions one may draw on the scales and mechanisms that drive the turbulence based on the measured $\zeta$.
Not only does $\zeta$ become insensitive from the influence of gravitational contraction with time, the non-weighted VSFs do also not reflect when the majority of kinetic energy has been transferred to small scales. 
Thus, it is highly important to clarify which definition of VSF is used and if it is indeed suitable for the respective study.
Yet, our analysis also shows that the principle of self-similarity relativises the deviations between the two approaches as the different orders of the VSFs keep scaling in the same way. 













