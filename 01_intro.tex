\section{Introduction}\label{intro}

It has long been known that star formation preferentially occurs within molecular clouds (MCs). 
However, the physics of the star formation process is still not completely understood.
It is obvious that gravity is the key factor for star formation as it drives collapse motions and operates on all scales.
However, one needs additional processes that stabilise the gas or terminate star formation quickly in order to explain the low star formation efficiencies observed in MCs. 
Although there are many processes that act at the different scales of MCs, turbulent support has often been argued to be the best candidate for this task.

In the literature, turbulence plays an ambiguous role in the context of star formation. 
In most of the cases, turbulence is expected to stabilise MCs on large scales \citep{Fleck1980,McKee1992,MacLow2003}, while feedback processes and shear motions heavily destabilise or even disrupt cloud-like structures \citep{Tan2013,Miyamoto2014}. 
However, it remains unclear whether there are particular mechanisms that dominate the driving of turbulence within MCs, as every process is supposed to be traced by typical features in the observables.
Yet, these features are either not seen or are too ambiguous to clearly reflect the dominant driving mode.
For example, turbulence that is driven by large-scale velocity dispersions during global collapse \citep{Ballesteros2011a,Ballesteros2011b,Hartmann2012} produces P-Cygni line profiles that have not yet been observed on scales of entire MCs. 
Internal feedback, on the contrary, seems more promising as it drives turbulence from small to large scales \citep{Dekel2013,Krumholz2014}.
Observations, though, demonstrate that the required driving sources need to act on scales of entire clouds; which typical feedback, such as radiation, winds, jets, or supernovae (SNe), cannot achieve \citep{Heyer2004,Brunt2009,Brunt2013}.

There have been many theoretical studies that have examined the nature and origin of turbulence within the various phases of the interstellar medium \citep[ISM;][and references within]{MacLow2004}. 
The most established characterisation of turbulence in general was introduced by \citet{Kolmogorov1941} who investigated fully developed, incompressible turbulence driven on scales larger than the object of interest, and dissipating on scales much smaller than those of interest.
In the scope of this paper this object is a single MC. 
MCs are highly compressible, though.
Only a few analytical studies have treated this case.
\citet{She1994} and \citet{Boldyrev2002}, for example, generalise and extend the predicted scaling of the decay of turbulence to supersonic turbulence.
\citet{Galtier2011} and \citet{Banerjee2013} provide an analytic description of the scaling of mass-weighted structure functions.

\citet{Larson1981} found a relation between the linewidth $\sigma$ and the effective radius $R$ of MCs.
Subsequent investigators have settled on the form of the relation being \citep{Solomon1987,Falgarone2009,Heyer2009}
\begin{equation} \label{eq:larson}
    \sigma \propto R^{1/2}.
\end{equation}
\citet{Goodman1998} showed that analysis techniques used to study this relation could be distinguished by whether they studied single or multiple clouds using single or multiple tracer species.
Explanations for this relation have relied on either turbulent cascades \citep{Larson1981,Kritsuk2013,Kritsuk2015,Gnedin2015,Padoan2016}, or the action of self-gravity \citep{Elmegreen1993,Vazquez2006,Elmegreen2007,Heyer2009,Ballesteros2011}.


These can potentially be distinguished by examining the velocity structure function.
\citet{Kritsuk2013} carefully reviews the argument for Larson's size-velocity relation depending on the turbulent cascade. 
In short, in an energy cascade typical for turbulence, the second-order structure function has a lag dependence $\ell^{\zeta(2)}$ with $\zeta(2) \simeq 1/2$. 
In \citet[hereafter \citetalias{IbanezMejia2016}]{IbanezMejia2016} the authors argued that uniform driven turbulence was unable to explain the observed relation in a heterogeneous interstellar medium, but that the relation could be naturally explained by hierarchical gravitational collapse.

In this paper, we examine the velocity structure functions of three MCs that formed self-consistently from SN-driven turbulence in the simulations by \citetalias{IbanezMejia2016} and \citet[][hereafter \citetalias{IbanezMejia2017}]{IbanezMejia2017}.

and study how the turbulence within the clouds' gas evolves.
The key questions we address are the following:
What dominates the turbulence within the simulated MCs? 
Does the observed linewidth-size relation arise from the turbulent flow?
How can structure functions inform us about the evolutionary state of MCs and the relative importance of large-scale turbulence, discrete blast waves, and gravitational collapse?

In Sect.~\ref{methods}, we introduce the simulated clouds in the context of the underlying physics involved in the simulations.
Furthermore, we describe the theoretical basics of velocity structure functions.
Sect.~\ref{results} demonstrates that the velocity structure function is a useful tool to characterise the dominant driving mechanisms of turbulence in MCs and can be applied to both simulated and observed data. 
We examine the influence of using one-dimensional velocity measurements, different Jeans refinement levels, density thresholds, and density weighting on the applicability of the velocity structure function and the results obtained with it in Sect.~\ref{discussion}.  
At the end of that section, we will also compare our results to observational studies.
We summarise our findings and conclusions in Sect.~\ref{conclusions}.
The simulation data and the scripts that this work is based on are published in the Digital Repository of the American Museum of Natural History \citep{Chira2018b}.




\endinput
