\section{Introduction}\label{intro}

It has long been know that star formation preferentially occurs in molecular clouds (MCs). 
Yet the exact set-up that leads to the on-set of star-forming activities is still not completely understood.
It is clear that gravity is the major factor as it drives collapse motions and operates on all scales.
However, one needs additional processes that stabilise the gas in order to explain the long life times of MCs, low star formation efficiencies and the formation of (filamentary) sub-structures observed in MCs. 
Although there are many processes that act on the different scales of MCs, turbulence is traditionally supposed to be the best candidate for this task.

In literature, turbulence has an ambiguous role in the context of star formation. 
In most of the cases, turbulence is expected to stabilise molecular clouds on large scales \citep{Fleck1980,McKee1992,MacLow2003}, while feedback processes and shear motions heavily destabilise or even disrupt cloud-like structures, offering a formation scenario for large filaments \citep{Tan2013,Miyamoto2014}. 
However, it is not entirely clear whether there are particular mechanisms that dominantly drive the turbulence with MCs, although all of them are supposed to show different imprints in the observables. 
For example, turbulence that is driven by large-scale velocity dispersions during global collapse \citep{Ballesteros2011a,Ballesteros2011b,Hartmann2012} produces P-Cygni lines. 
Despite that such lines are normally not observed, this process cannot explain the long lifetimes of giant molecular clouds. 
Internal feedback, contrary, seems more promising as it drives turbulence outwards \citep{Dekel2013,Krumholz2014}.
Observations, though, demonstrate that the required driving sources need to act on scales of entire clouds; which typical feedback processes cannot achieve \citep{Brunt2009,Brunt2013,Heyer2004}.

There have also been many theoretical studies that have examined the nature and origin of turbulence within the phases of the ISM \citep[and references within]{MacLow2004}. 
The most established work has been conducted by \citet{Kolmogorov1941} who investigated fully developed, incompressible turbulence that is driven on scales larger than the object of interest.
In the scope of this paper this object is a single molecular cloud. 
The underlying assumptions of \citeauthor{Kolmogorov1941}'s describe the very special scenario of a divergence-free velocity field. 
This is not necessarily the case in a MC, yet analytical studies without these assumptions are still rare, but exist. 
\citet{She1994} and \citet{Boldyrev2002}, for example, generalise and extend the predicted scaling of the decay of turbulence to supersonic turbulence.
\citet{Galtier2011} and \citet{Banerjee2013} provide an analytic description of the scaling of mass-weighted structure functions.

In this paper, we examine three molecular clouds that formed self-similarly in the simulations by \citet[\citetalias{IbanezMejia2016} and \citetalias{IbanezMejia2017}, hereafter]{IbanezMejia2016,IbanezMejia2017} and study how the turbulence within the clouds' gas evolves.
The key questions we answer are the following: 
What dominates the turbulence within the simulated molecular clouds? 
Is there a method that can trace the dominant modes of turbulence driving reliably, also in observational studies?

In Sect.~\ref{methods}, we introduce the simulated clouds in the context of the underlying physics involved in the simulations.
Furthermore, we describe the theoretical basics of velocity structures functions.
Sect.~\ref{results} demonstrates that the velocity structure function is a useful tool to characterise the dominant driving mechanisms of turbulence in molecular clouds and can be applied on both simulated and observed data. 
We also discuss the influences of utilising one-dimensional velocity measurements, different Jeans refinement levels, density thresholds and density weighting on the applicability of the velocity structure function and the results obtained with it in Sect.~\ref{discussion}.  
We summarise our findings and conclusions in Sect.~\ref{conclusions}.



\endinput
