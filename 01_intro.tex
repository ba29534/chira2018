\section{Introduction}\label{intro}

It has long been know that star formation preferentially occurs in molecular clouds, yet the exact set-up that leads to the on-set of star-forming activities is still not completely understood.
It is clear that gravity is the major factor as it drives collapse motions and operates on all scales.
However, one needs additional processes that counteract against gravity in order to explain the long life times of molecular clouds, low star formation efficiencies and the formation of (filamentary) sub-structures observed in molecular clouds. 
Although there are many processes acting on the scales of molecular clouds, turbulence are traditionally supposed to be the best candidate for efficiently counteract against gravitational collapse.

In literature, turbulence has an ambiguous role in the context of star formation. 
In most of the cases, turbulence is expected to stabilise molecular clouds on large scales \citep{Fleck1980,McKee1992,MacLow2003}, while feedback processes and shear motions heavily destabilise or even disrupt cloud-like structures, offering a formation scenario for large filaments \citep{Tan2013,Miyamoto2014}. 

Yet, it is not entirely clear which mechanisms drive the turbulence within molecular clouds dominantly although all of them are supposed to show different imprints in the observables. 
For example, turbulence that is driven by large-scale velocity dispersions during global collapse \citep{Ballesteros2011a,Ballesteros2011b,Hartmann2012} produces P-Cygni lines. 
These lines are, though, normally not observed, and this mechanism cannot explain the long lifetimes of giant molecular clouds. 
The scenario in which internal feedback sources drive turbulence outwards \citep{Dekel2013,Krumholz2014} seems more promising. 
However, observations demonstrate that the required driving sources need to act on scales of entire clouds, which typical feedback processes cannot achieve \citep{Brunt2009,Brunt2013,Heyer2004}.

There have also been many theoretical studies examining the nature and origin of turbulence \citep[and references within]{MacLow2004}. 
The most established work has been conducted by \citet{Kolmogorov1941} who investigated fully developed, incompressible turbulence that is driven on scales larger than the object of interest; in the scope of this paper this is a single molecular cloud. 
The underlying assumptions, however, describe the very special scenario of a divergence-free velocity field. 
Analytical studies without these assumptions are still rare, but exist. 
\citet{She1994} and \citet{Boldyrev2002}, for example, generalise and extend the predicted scaling of the decay of turbulence to supersonic turbulence.
\citet{Galtier2011} and \citet{Banerjee2013} provide an analytic description of the scaling of mass-weighted structure functions.

In this paper, we examine three molecular clouds that formed self-similarity in the simulations by \citet[\citetalias{IbanezMejia2016}, hereafter]{IbanezMejia2016} and study the evolution of turbulence as traced by the gas within the model clouds. 
The key questions we answer are the following: 
What dominates the turbulence within the simulated molecular clouds? 
Is there a method that can trace the dominant modes reliably, also in observational studies?

In Sect.~\ref{methods}, we introduce the model clouds in the context of the underlying simulations, as well as the theoretical basics of velocity structures functions.
Sect.~\ref{results} demonstrates that velocity structure functions are a useful tool to characterise the dominant driven source of turbulence in molecular clouds and can be applied on both simulated and observed data. 
Furthermore, we also discuss the influences of utilising one-dimensional velocity measurements, different Jeans refinement levels, density thresholds and density weighting have on the applicability of velocity structure functions and the results obtained with them.  
We summarise our findings and conclusions in Sect.~\ref{conclusions}.



\endinput