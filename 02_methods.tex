\section{Methods}\label{methods}


\subsection{Cloud models}\label{methods:clouds}

The analysis in this paper is based on a sample of three molecular clouds (MCs) found within a three-dimensional (3D), magnetohydrodynamical, adaptive mesh refinement simulation using the FLASH code \citep{Fryxell2000}.  
\citetalias{IbanezMejia2016} and \citetalias{IbanezMejia2017}, as well as \citet[\citetalias{Chira2018} hereafter]{Chira2018}, describe the simulations and the clouds in more detail. 
We summarise the most relevant properties. 

The numerical simulation models a $1\times1\times40$~kpc$^3$ section of the multi-phase, turbulent ISM of a disk galaxy, where dense structures form self-consistently in convergent, turbulent flows \citepalias{IbanezMejia2016}.  
The model includes gravity---a background galactic-disk potential accounting for a stellar component and a dark matter halo, as well as self-gravity turned on after 250~Myr of simulated time---supernova-driven turbulence, photoelectric heating and radiative cooling, and magnetic fields. 
Although hundreds of dense clouds form within the simulated volume, \citetalias{IbanezMejia2017} focused on three clouds, which were re-simulated at a much higher spatial resolution, which we map to a uniform grid of 0.1 pc zones.
In this paper, we use the data within (40~pc)$^{3}$ subregions centred in the high-resolution clouds' centres of mass.
The internal structures of MCs are resolved using adaptive mesh refinement, focusing grid resolution on dense regions where Jeans unstable structures must be resolved with a minimum of 4 cells ($\lambda_J > 4~\Delta x_{\rm min}$).
For a maximum resolution of $\Delta x = 0.1$~pc, the corresponding maximum resolved density is $8~\times 10^3$~cm$^{-3}$ for gas at a temperature of 10~K \citepalias[e.g.][Eq.~15]{IbanezMejia2017}.
The MCs have initial masses at the onset of self-gravity of $3~\times 10^3$, $4~\times 10^3$, and $8~\times 10^3$~M$_{\odot}$ and are denoted as \texttt{M3}, \texttt{M4}, and \texttt{M8}, respectively, hereafter.

However, it is important to point out that the clouds are embedded within a complex turbulent environment, gaining and losing mass as they evolve.
\citetalias{IbanezMejia2017} described the time evolution of the properties of all three clouds in detail, in particular, mass, size, velocity dispersion, and accretion rates, in the context of MC formation and evolution within a galactic environment.
\citetalias{Chira2018} studied the properties, evolution, and fragmentation of filaments that self-consistently condense within these clouds. 
They paid particular attention to the accuracy of typical stability criteria for filaments, comparing their measurements to the theoretical predictions, showing that theoretical models do not capture the complexity of fragmentation due to their simplifying assumptions.


\subsection{Velocity Structure Function}\label{methods:vsf}

In this paper, we probe the power distribution of turbulence throughout the entire simulated MCs by using the velocity structure function (VSF).
The VSF is a two-point correlation function,
\begin{equation}
	{S}_p (\ell) = \langle \, |\Delta \vec{v}|^p  \, \rangle
	\label{equ:method:def_vsf}
\end{equation}
that measures the mean velocity difference to the $p^\mathrm{th}$ power
\begin{equation}\Delta \vec{v} (\vec{\ell}) = \vec{v}(\vec{x}+\vec{\ell}) - \vec{v}(\vec{x})
\end{equation} 
between two points $\vec{x}$ and $\vec{x}+\vec{\ell}$, with $\vec{\ell}$ being the direction vector pointing from the first to the second point.  
$S_p$ is usually reported as a function of lag distance, $\ell = |\vec{\ell}|$, between the correlated points.
The VSF estimates the occurrence of symmetric motions (e.g., rotation, collapse, outflows), as well as rare events in random turbulent flows (e.g., supernovae) in velocity patterns.
Those patterns become more prominent the higher the value of $p$ \citep{Heyer2004}.
\citet{Padoan2016a} defined a density-weighted velocity structure function
\begin{equation}
	{S}_p (\ell) = \frac{\langle \, \rho(\vec{x}) \rho(\vec{x}+\vec{\ell}) \, |\Delta \vec{v}|^p  \, \rangle}{\langle  \, \rho(\vec{x}) \rho(\vec{x}+\vec{\ell}) \, \rangle}.
	\label{equ:method:def_vsf_dw}
\end{equation}
In many cases, though, a three-dimensional computation of the VSF cannot be performed because of the observational constraint that only line-of-sight velocities are available.
We will therefore compare our three-dimensional (3D) results to one-dimensional (1D), density-weighted VSFs
\begin{equation}
	\mathit{S}_{p,\mathrm{1D}} (\ell) = \frac{\langle \, \rho(\vec{x}) \rho(\vec{x}+\vec{\ell}) \, |\Delta 
        \vec{v} \cdot \vec{e}_i|^p  \, \rangle}{\langle  \, \rho(\vec{x}) \rho(\vec{x}+\vec{\ell}) \, \rangle} ,
	\label{equ:results:def_vsf_1d}
\end{equation}
with $\vec{e}_i$ representing the unit vector along the $i$~=~$x$-, $y$-, or $z$-axis.

The first several orders of the VSF have a physical meaning. 
For example, $\mathit{S}_1$ gives the mean of relative velocities between any two points reflecting the modes created by different gas flows, while $\mathit{S}_2$ is proportional to the kinetic energy, making it a good probe of how the turbulent energy is distributed among different scales.

For fully developed turbulence the VSF is well-described by a power-law relation \citep{Kolmogorov1941,She1994,Boldyrev2002}:
\begin{equation}
	\mathit{S}_p (\ell) \propto \ell^{\zeta(p)}.
	\label{equ:method:propto_zeta}
\end{equation}
Note that the scaling exponent of that power-law relation, $\zeta$, depends on many parameters, such as the order of the VSF, as well as the properties and composition of the studied turbulent flow, such as its geometry, compressibility, or magnetization.
Many studies of VSFs distinguish between longitudinal and transverse velocity components, or compressible and solenoidal components, because those are expected to behave differently, especially towards larger lag distances \citep{Gotoh2002,Schmidt2008,Benzi2010}.
However, the differences are mostly negligible on the scales we focus on. 
Since these decompositions are also hard to perform on observational data, we focus only on the total VSF.

Theoretical studies predict values of $\zeta$ depending on the nature of the turbulence and the order $p$.
\citet{Kolmogorov1941} predicts that the third-order exponent, $\zeta(3)$, is equal to unity for an incompressible flow.
This results in the commonly known prediction that the kinetic energy decays with $E_{\mathrm{kin}}(k) \propto k^{-\frac{5}{3}}$, with $k = \frac{2 \pi}{\ell}$ being the wavenumber of the turbulence mode.

For a supersonic flow, however, $\zeta(3) >1$ is expected.
Based on \citeauthor{Kolmogorov1941}'s work, \citet{She1994} and \citet{Boldyrev2002} extended and generalised the analysis and predicted the following.
For an incompressible filamentary flow \citet{She1994} predict that
the VSFs scale with power law index
\begin{equation}
	\zeta(p) = \frac{p}{9} + 2 \left[ 1 - \left( \frac{2}{3} \right)^{\frac{p}{3}} \right] ,
	\label{equ:method:she}
\end{equation}
while supersonic flows with sheet-like geometry are predicted to scale with \citep{Boldyrev2002}
\begin{equation}
	\zeta(p) = \frac{p}{9} + 1 - \left( \frac{1}{3} \right)^{\frac{p}{3}}.
	\label{equ:method:boldyrev}
\end{equation}
\citet{Benzi1993} introduced the principle of extended self-similarity.
It proposes that there is a constant relationship between the scaling
exponents of different orders at all lag scales so that $\zeta$ can be measured from $S_p/S_3$, which typically gives much clearer power-law behaviour.
The self-similarity parameter is defined as,
\begin{equation}
	Z(p) = \frac{\zeta(p)}{\zeta(3)}.
	\label{equ:method:z_def}
\end{equation} 
Since the predicted values for $\zeta(p)$ by \citet{She1994} and \citet{Boldyrev2002} are normalised so that $\zeta(3)$~=~1, Eq.~(\ref{equ:method:she}) and~(\ref{equ:method:boldyrev}) also provide the predictions for $Z(p)$, respectively.

For the discussion below, we measure $\zeta$ by fitting a power-law,
given by
\begin{equation}
	\log_{10}\left[ S_p(\ell) \right] = \log_{10}\left(A\right) + \zeta \log_{10}(\ell) ,
	\label{equ:method:fitting}
\end{equation}
with $A$ being the proportionality factor of the power-law to the simulated measurements.

In order to reduce the computational effort, we divide the range of 3D lag distances, $|\vec{\ell}|$, into 40 equidistantly separated bins ranging from~0.1 to 30~pc.
This means that the measurements at the given lag interval $\ell_i$ we will show below are based on data with lag distances $\ell_{i-1} < \ell \leq \ell_i$.

Furthermore, we reduce the computational effort of this study generally focussing on clouds, defined as regions above a fixed number density threshold with fiducial value $n_{\mathrm cloud} = 100$~cm$^{-3}$.  We have chosen this threshold as it roughly corresponds to the density when CO becomes detectable.
However, \citetalias{IbanezMejia2017} shows that there is usually no sharp increase in density between the ISM and the clouds. 
Instead, the gas becomes continuously denser towards the centres of mass within the clouds. 
Consequently, our use of a density threshold is a somewhat artificial boundary between the clouds and the ISM. Observationally, however, introducing a column density (or intensity) threshold is unavoidable, be it due to technical limitations (e.g., detector sensitivity) or the nature of the underlying physical processes (for example, excitation rates, or critical densities).
Therefore, it is important to study how a density threshold influences the VSF and its evolution.

At our fiducial density threshold, we actually consider only $\leq$1.5\% of the volume in the high resolution cube.
To understand the influence of this limitation we set up a test scenario (see Sect.~\ref{results:densthres}) by removing the density threshold, by setting n$_\mathrm{cloud}$~=~0~cm$^{-3}$, that results in analysing the entire data cube.
As this would be too computationally expensive, we randomly choose a set of 5\% of the total number of cells as reference points and compute relative velocities to these cells only.
We emphasise that this does not mean that we only calculate the relative velocities between these cells.
Rather this subsample of cells represent the starting vectors $\vec{x}$ to which the velocities of all other cells $\vec{x} + \vec{\ell}$ in the same cube are compared to.
This way we reduce the risk of ignoring or emphasising any spatial direction or angle.

Note that by choosing the starting points randomly we ensure that all parts of the cubes are considered. 
As a consequence, there is only a small likelihood (5\%~$\times$~1.5\%~=~0.075\%) that any given cell chosen will be in the cloud.
Therefore, we emphasise that it is likely that the two subsamples (no density threshold and cloud-only) do not have a common subset of starting vectors.
Nevertheless, the random sample still includes $>4 \times 10^3$ cells in the cloud, so the sample does include information on VSFs of material in the cloud.

\endinput
