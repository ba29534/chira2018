\section{Summary \& Conclusions}\label{conclusions}

In this paper, we analyse the \textbf{VSFs} of molecular clouds that have formed within 3D adaptive mesh FLASH simulations of the self-gravitating, magnetised, SN-driven ISM by \citetalias{IbanezMejia2016}.
The main results are as follows.

\begin{itemize}
\item \textbf{The scaling of VSFs depends on both internal turbulence driving such as gravitational contraction, and external driving such as external SN blast waves. We find that the power-law scaling $\zeta$ of 3D VSFs reflects the development of gravitational contraction, while the extended self-similarity scaling $Z$ reveals interactions of clouds with large-scale flows and blast waves.}
\item As long as the molecular cloud is not affected by a shock, $Z$ \textbf{agrees well} with predicted values for supersonic flows, \textbf{even as gravitational collapse proceeds.}
\item We tested the influence of Jeans refinement on the VSFs. We find that the absolute amount of kinetic energy does not influence the evolution of $\zeta$ and $Z$, \textbf{but that better resolution of external shocks can produce changes in both quantities}.
\item \textbf{Comparison of 3D and 1D VSFs shows differences in detail, but qualitative agreement in behaviour of both $\zeta$ and $Z$, except when strong transverse flows dominate the velocity field. Thus, observed 1D VSFs can be useful diagnostics. }
\item \textbf{We calculated cloud VSFs using a density threshold to isolate the cloud material, as would characteristically happen in an observation of molecular material. Without such a threshold, our VSFs are dominated by the diffuse ISM. The extended self-similarity scaling $Z$ lies just below the value predicted for incompressible turbulence by \citet{She1994}. This is consistent with the low Mach number in the hot, diffuse, ISM filling most of the volume of our simulation.}
\item We investigate the influence of defining the VSF with and without density weighting. We find that the qualitative behaviour is traced by both approaches. However, the scaling of the non-weighted VSF $\zeta$ is always positive, \textbf{not falling nearly as far as for the} density-weighted VSF. \textbf{The density-weighted VSF reflects the kinetic energy distribution better as gravitational collapse proceeds to smaller and smaller scales. (Note that in, for example, CO observations, optical depth effects may obscure this behaviour.) }
\item \textbf{We compared our results with measurements of both $\zeta$ and $Z$ in observational studies. We see that our findings agree well with observations within periods in which the clouds' flows are dominated by global gravitational contraction and strong structure formation, as well as starting fragmentation; which reflects the conditions of embedded star formation activities within the observed molecular clouds.}
\end{itemize}

Our analysis shows that VSFs are \textbf{useful} tools for examining the driving source of turbulence within molecular clouds.
However, studies that use VSFs need to precisely review the assumptions and parameters \textbf{included} in their analysis as those can have a significant influence on the results.

For our simulated clouds, the VSFs illustrate that gravitational contraction dominates the evolution of the clouds.
During contraction, the VSF scaling parameter $\zeta(p)$ drops in value and can even become negative as kinetic energy concentrates on small scales.
Nevertheless, the extended self-similarity scaling parameters $Z(p)$ continue to agree with the analytic prediction for compressible turbulence except for short periods during which SN blast waves \textbf{increase power on multiple scales.
Because such blast waves are neither homogeneous nor isotropic, they often lead to transient non-power law scaling of the VSFs, and thus strong departures from uniform turbulent behaviour of $Z(p)$.} 



\endinput
