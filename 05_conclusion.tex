\section{Summary \& Conclusions}\label{conclusions}

In this paper, we analyse the VSFs of MCs that have formed within 3D magnetohydrodynamical, adaptive mesh refinement, FLASH simulations of the self-gravitating, SN-driven ISM by \citetalias{IbanezMejia2017}, including both density weighting and a density cutoff.
The main results are as follows.

\begin{itemize}
	\item The scaling of the density-weighted VSFs depends on a combination of turbulence and more coherent processes such as SN blast wave impacts and gravitational contraction.
    We find that the power-law scaling $\zeta$ of 3D density-weighted VSFs reflects the development of gravitational contraction, while the extended self-similarity scaling $Z$ reveals interactions of clouds with large-scale blast waves.
    \item The two different proposed explanations for Larson's size-velocity relationship, a turbulent cascade and gravitational contraction,  appear to apply to different stages in the evolution of MCs, as well as different observational techniques. It appears {\em coincidental} that they have the same functional relationship of length to velocity, which has led to confusion of one with the other.
    \begin{itemize}
        \item MCs dominated by uniform turbulence show a second-order VSF with $\zeta(2) \simeq 1/2$.  The same result can be found for clouds undergoing strong gravitational contraction by computing the VSF without density weighting, which is dominated by the low-density, turbulence-dominated outer regions of clouds.
    \item Examining the overall velocity dispersion of gravitationally dominated clouds undergoing star formation, on the other hand, reflects the dynamics of gravitational collapse.  In this case, the cloud shows a shallow VSF dependence $\zeta(2) \lesssim 0$. This reflects strong flows at small scales. However, such gravitationally contracting clouds were shown by \citetalias{IbanezMejia2016} to have an overall square-root velocity-radius relationship (Eq.~[\ref{eq:larson}]) given by free-fall or virial equilibrium (which \citealt{Ballesteros2011} note differ by only $2^{1/2}$).
    \end{itemize}
	\item As long as the MC is not affected by a shock, $Z$ agrees well with predicted values for supersonic flows, even as gravitational collapse proceeds.
	\item We test the influence of Jeans refinement on the VSFs. We find that the absolute amount of kinetic energy does not influence the evolution of $\zeta$ and $Z$, but that better resolution of external shocks can produce changes in both quantities.
	\item Comparison of 3D to 1D VSFs shows differences in detail, but qualitative agreement in the behaviour of both $\zeta$ and $Z$, in particular when gravity dominates gas dynamics. Thus, observed 1D VSFs can be useful diagnostics in gravitationally bound and contracting regions. On the other hand, differences arise when strong transverse flows dominate the velocity field. 
	\item We calculate cloud VSFs using a density threshold to isolate the cloud material, as would characteristically happen in an observation of molecular material. Without such a threshold, our VSFs are dominated by the diffuse ISM. In that case, the extended self-similarity scaling $Z$ lies just below the value predicted for isotropic, incompressible turbulence by \citet{She1994}. This is consistent with the low Mach number in the hot, diffuse, ISM filling most of the volume of our simulation.
	\item We investigate the influence of defining the VSF with and without density weighting. We find that the qualitative behaviour is traced by both approaches. However, the scaling of the non-weighted VSF $\zeta$ is always positive, not falling nearly as far as for the density-weighted VSF. The density-weighted VSF reflects the kinetic energy distribution better as gravitational collapse proceeds to smaller and smaller scales. (Note that in, for example, CO observations, optical depth effects may obscure this behaviour.) 
	\item We compare our results with measurements of both $\zeta$ and $Z$ in observational studies. We see that our findings agree well on average with observations within periods during which the clouds' flows are dominated by global gravitational contraction and strong structure formation, as well as starting fragmentation. This reflects the conditions of embedded star formation activity within observed MCs.
\end{itemize}

Our analysis shows that VSFs are useful tools for examining the driving source of turbulence within MCs.
However, studies that use VSFs need to precisely review the assumptions and parameters included in their analysis as those can have a significant influence on the results.

For our simulated clouds, the VSFs illustrate that gravitational contraction dominates the evolution of the clouds.
During contraction, the VSF scaling parameter $\zeta(p)$ drops in value and can even become negative as kinetic energy concentrates on small scales.
Nevertheless, the extended self-similarity scaling parameters $Z(p)$ continue to agree with the analytic prediction for compressible turbulence except for short periods during which SN blast waves increase power on multiple scales.
Because such blast waves are neither homogeneous nor isotropic, they often lead to transient non-power law scaling of the VSFs, and thus strong departures from uniform turbulent behaviour of $Z(p)$.


\endinput
