\section{Summary \& Conclusions}\label{conclusions}

In this paper, we analyse the turbulent structures of molecular clouds and filaments that have formed within 3D AMR FLASH simulations of the self-gravitating, magnetised, supernova-driven ISM by \citet{IbanezMejia2016}.
The main results are as follows.

\begin{itemize}
	\item The scaling of velocity structure functions (VSFs) is sensitive to both internal (gravitational contraction) and external (SN shocks, winds) driving sources of turbulence. Applied on simulated data, the time evolution of the scaling exponent, $\zeta$, can reveal which driving mechanism dominates the turbulence of an entire molecular cloud. The self-similarity parameter, $Z$, though, is not directly sensitive to gravitational contraction. Yet, it can be used as observational tracer as it significantly reacts to SNe and winds.
	\item As long as the molecular cloud is not affected by a shock, $Z$ is in good agreement with predicted values for supersonic flows. This makes it a fine probe for the properties of dominant turbulent modes, such as the geometry, and their evolution in the context with the evolution of the cloud. 
	\item We test the influence of Jeans refinement on the VSFs. We find that the absolute amount of kinetic energy does not influence the evolution of $\zeta$ and $Z$, as long as the power spectrum is properly resembled, or similarly resembled by the compared samples.
	\item We see that the behaviour of the VSFs can be tracked with similar results based on 3D (i.e., from simulated data) and 1D (i.e., from observational data) velocity information, as long as there is no dominating flow driving the gas into a direction perpendicular to the line of sight. In the general case of a fully developed turbulent field, the both $\zeta$ and $Z$ evolve similarly in both scenarios, even though the actual values might not be similar.
	\item We test the influence of introducing a density threshold on the VSFs. We see significant differences both quantitatively and qualitatively. The VSFs that based on the unfiltered data are much steeper than the cloud-only VSFs and do not reflect any interaction with any of the driving sources, including SN shocks. The measured $Z$ values are constant in time and and for all clouds. This means that the turbulence we examine in this sub-project reflects the ISM in our entire galactic-scale simulations. The values of $Z$ are slightly below the value predicted for a filamentary flow by \citet{She1994}. We conclude that the turbulence in the modelled ISM consists of vortices that are similar to filamentary flows, yet with a ratio of average length scale of the two moments being more equal to unity than in the filamentary case.
    \item We investigate the influence of defining the velocity structure function with and without taking density weighting into account. We see that the general, qualitative behaviour is traced by both ansatzes in the same way. In the case of the non-weighted VSF the scaling exponents are always positive and evolve flatter than they do for the density-weighted VSF.
\end{itemize}

Our analysis shows that VSFs are fine tools for examining the driving source of turbulence within molecular clouds.
Therefore, we recommend its usage in future studies of molecular clouds.
However, studies that utilise VSFs need to precisely review the assumptions and parameters they imply in their analysis as those can have a significant influence on the out-coming results.

For the model clouds, the VSFs illustrate that gravitational contraction dominates the evolution of the clouds for most their evolution, with short periods within which SN shock waves accelerate the turbulent powers on all scales. 
Yet it requires further studies to verify this to be the common fragment formation scenario. 
Especially, a higher Jeans length refinement is needed to resolve the velocity structures on scales of individual grid cells (0.1~pc in this case).
This is crucial for following the local behaviour of the gas as neither the average behaviour of the filaments nor the dominant turbulence driving source of the entire molecular clouds mirror the underling flow patterns that are necessary for this scenario. 



\endinput
