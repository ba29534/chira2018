\section{Methods}\label{methods}


\subsection{Cloud models}\label{methods:clouds}

%\rc{[general comment: In this sub-section I basically copied the descriptions we have used in the previous paper. I am not sure whether the summary given here in to short as I also wanted to provide a summary of the work we have done with the data already. I would appreciate what you think about the structure of this subsection.]}

The analysis in this paper is based on a sample of three molecular clouds (MCs) found within a three-dimensional, 
%mm  [little used acronyms] (3D) 
magnetohydrodynamical,
%mm~(MHD), 
adaptive mesh refinement
%mm ~(AMR) numerical 
simulation by \citet{IbanezMejia2016}, using the FLASH code \citep{Fryxell2000}.
%mm [already defined] \citet[hereafter \citetalias{IbanezMejia2016} and \citetalias{IbanezMejia2017}, respectively]
\citetalias{IbanezMejia2016,IbanezMejia2017}, and \citet[\citetalias{Chira2018} hereafter]{Chira2018} describe the simulations and the clouds in more detail. 
We summarise the most relevant properties. 

The numerical simulation models a 
%mm
   $1\times1\times40$~kpc 
section of the multi-phase, turbulent ISM of a disk galaxy, where dense structures form self-consistently in convergent, turbulent flows \citepalias{IbanezMejia2016}.  
The model includes gravity---a background galactic-disk potential accounting for a stellar component and a dark matter halo, as well as self-gravity turned on after 250~Myr of simulated time---SN-driven turbulence, photoelectric heating and radiative cooling, and magnetic fields. 
Although hundreds of dense clouds form within the simulated volume, \citetalias{IbanezMejia2017} focused on three clouds, which were re-simulated at a much higher spatial resolution,
%mm
    which we map to a uniform grid of 0.1 pc zones.
In this paper, we use the data within (40~pc)$^{3}$ subregions centred in the high-resolution clouds' centres of mass.
%The three MCs of our sample are (40~pc)$^{3}$ subregions of the entire $1~\times~1~\times~40$~kpc$^3$ volume.
%The authors of \citetalias{IbanezMejia2017} have re-simulated the clouds with an effective spatial resolution of $\Delta x_{\rm min}=0.1$~pc when mapped onto a $400\times 400~\times 400$ grid cells containing cube, respectively.
The internal structures of MCs are resolved using adaptive mesh refinement, focusing grid resolution on dense regions where Jeans unstable structures must be resolved with a minimum of 4 cells ($\lambda_J > 4~\Delta x_{\rm min}$).
For a maximum resolution of $\Delta x = 0.1$~pc, the corresponding maximum resolved density is $8~\times 10^3$~cm$^{-3}$ for gas at a temperature of 10~K \citepalias[e.g.][Eq.~15]{IbanezMejia2017}.
This means that we can trace fragmentation down to 0.4~pc, but cannot fully resolve objects that form at smaller scales.
The MCs have initial masses at the onset of self-gravity of $3~\times 10^3$, $4~\times 10^3$, and $8~\times 10^3$~M$_{\odot}$ and are denoted as \texttt{M3}, \texttt{M4}, and \texttt{M8}, respectively, hereafter.

%mm [moved para break up]
However, it is important to point out that the clouds are embedded within a complex turbulent environment, gaining and losing mass as they evolve.
%
%mm This set-up opens opportunities for many different kind of studies. The authors of
\citetalias{IbanezMejia2017} described the time evolution of the properties of all three clouds in detail, in
particular, 
%mm the authors investigated typical properties of the clouds, such as
mass, size, velocity dispersion, and accretion rates, in the context of MC formation and evolution within a galactic environment.
\citetalias{Chira2018} studied the properties, evolution, and fragmentation of filaments that self-consistently condense within these clouds. 
They paid particular attention to the accuracy of typical stability criteria for filaments, comparing their measurements to the theoretical predictions, showing that theoretical models do not capture the complexity of fragmentation due to their simplifying assumptions.

In this paper, we focus on the driving sources of turbulence within the simulated MCs and the signatures they imprint on 
%mm observables, such as [anything else?]
the velocity structure function (VSF).


\subsection{Velocity Structure Function}\label{methods:vsf}

We probe the power distribution of turbulence throughout the entire simulated MCs by using the VSF.
The VSF is a two-point correlation function 
%mm [insert definition]
\begin{equation}
	{S}_p (\ell) = |\Delta \vec{v} (\vec{\ell})|^p ,
    \label{equ:method:def_vsf}
\end{equation}
that measures the mean velocity difference, $\Delta \vec{v} (\vec{\ell}) = \vec{v}(\vec{x}+\vec{\ell}) - \vec{v}(\vec{x})$, between two points $\vec{x}$ and $\vec{x}+\vec{\ell}$ (with $\vec{\ell}$ being the direction vector pointing from the first to the second point), to the $p^\mathrm{th}$ power as function of lag distance, $\ell = |\vec{\ell}|$, between the correlated points.
The VSF estimates the occurrence of symmetric motions (e.g., rotation, collapse, outflows), as well as rare events in random turbulent flows (e.g., SNe) in velocity patterns.
Those patterns become more prominent the higher the order $p$ is \citep{Heyer2004}.

%mm For data on an Eulerian grid, such as ours, the density-weighted definition of the VSF, $\mathit{S}_p$, is given by
\citet{Padoan2016a} defined a density-weighted structure function
\begin{equation}
	{S}^d_p (\ell) = \frac{\langle \, \rho(\vec{x}) \rho(\vec{x}+\vec{\ell}) \, |\Delta \vec{v}|^p  \, \rangle}{\langle  \, \rho(\vec{x}) \rho(\vec{x}+\vec{\ell}) \, \rangle}.
    \label{equ:method:def_vsf_dw}
\end{equation}
%mm With this definition one can see that each order of the VSF has
          The first several orders of the VSF have 
a physical meaning. 
For example, $\mathit{S}_1$
%mm is to the mean relative velocities between \textbf{two cells}, [is this what you meant?] 
     gives the mean of relative velocities between any two points 
reflecting the modes created by different gas flows, while
$\mathit{S}_2$ is proportional to the kinetic energy, making it a good probe of how the turbulent energy is
%mm transported to
    distributed among 
different scales.

\textbf{For fully developed} turbulence 
%mm is fully developed 
the VSF is 
%mm supposed to be 
well-described by a power-law relation \citep{Kolmogorov1941,She1994,Boldyrev2002}:
\begin{equation}
	\mathit{S}_p (\ell) \propto \ell^{\zeta(p)} .
    \label{equ:method:propto_zeta}
\end{equation}
Note that the scaling exponent of that power-law relation, $\zeta$, depends on many parameters, such as the order of the VSF, as well as the properties and composition of the studied turbulent flow, such as its geometry, compressibility, or 
%mm Mach number. [redundant with compressibility]
      magnetization.
Many studies of VSFs distinguish between longitudinal and transverse velocity components, or compressible and solenoidal components, because those are expected to behave differently, especially towards larger lag distances \citep{Gotoh2002,Schmidt2008,Benzi2010}.
However, the differences are mostly negligible on the scales we focus on. 
%mm This and the fact that those components are observationally very hard to differentiate are the reasons why we analyse all components in a common sample.
    Since these decompositions are also hard to perform on observational data, we focus only on the total VSF.

Theoretical studies predict values of $\zeta$ depending on the nature of the turbulence and the order $p$.
For example, \citet{Kolmogorov1941} predicts that the third-order exponent, $\zeta(3)$, is equal to unity for an incompressible
%mm , transonic [a transonic flow is highly compressible by the standards of the turbulence community!]
flow.
This results in the commonly known prediction that the kinetic energy decays with $E_{\mathrm{kin}}(k) \propto k^{-\frac{5}{3}}$, with $k = \frac{2 \pi}{\ell}$ being the wavenumber of the turbulence mode.

For a supersonic flow, however, 
%mm it is always supposed to be greater or equal to unity.
       $\zeta(3) >1$ is expected.
Based on \citeauthor{Kolmogorov1941}'s work, \citet{She1994} and \citet{Boldyrev2002} extended and generalised the analysis and predicted the following.
For an incompressible filamentary flow \citet{She1994} predict that the VSFs scale with,
\begin{equation}
	\zeta(p) = \frac{p}{9} + 2 \left[ 1 - \left( \frac{2}{3} \right)^{\frac{p}{3}}, \right] %mm = Z_\mathrm{She}(p) ,
    \label{equ:method:she}
\end{equation}
while supersonic flows with sheet-like geometry are supposed to scale with \citep{Boldyrev2002},
\begin{equation}
	 \zeta(p) = \frac{p}{9} + 1 - \left( \frac{1}{3} \right)^{\frac{p}{3}}.
%mm= Z_\mathrm{Boldyrev}(p) .
    \label{equ:method:boldyrev}
\end{equation}
\citet{Benzi1993} introduced the principle of extended self-similarity.
It proposes that there is a constant relationship between the scaling exponents of different orders at all lag scales. 
%mmNormally, the self-similarity parameter, $Z$, normally refers to the 3$^\mathrm{rd}$ order VSF and is given by:
      The self-similarity parameter
\begin{equation}
	Z(p) = \frac{\zeta(p)}{\zeta(3)} .
	\label{equ:method:z_def}
\end{equation} 
Since \textbf{the predicted values for $\zeta(p)$ by \citet{She1994} and \citet{Boldyrev2002}} are normalised so that $\zeta(3)$~=~1, Eq.~(\ref{equ:method:she}) and~(\ref{equ:method:boldyrev}) also provide the predictions for $Z(p)$, respectively.

For the discussion below, we measure $\zeta$ by fitting a power-law, given by
\begin{equation}
	\log_{10}\left[ S_p(\ell) \right] = \log_{10}\left(A\right) + \zeta \log_{10}(\ell) ,
    \label{equ:method:fitting}
\end{equation}
with $A$ being the proportionality factor of the power-law to the simulated measurements.
For the calculations, we only take cells with a minimal number density of 100~cm$^{-3}$ into account as this threshold defines the volume of the clouds.
%mm For reducing
    In order to reduce
the computational effort, we divide the range of three-dimensional lag distances, $|\vec{\ell}|$, into 40 equidistantly separated bins ranging from~0.1 to 30~pc.
%mm ************ 
{\bf [there must be more to your optimization method than this.  Fill in the details?]}
% *************************
This means that the measurements at the given lag interval $\ell_i$ we will show below are based on data with lag distances $\ell_{i-1} < \ell \leq \ell_i$.

%mm [The material following this point is results, not methods.  I have therefore moved it into the next section.]


\endinput
