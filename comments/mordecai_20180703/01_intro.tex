\section{Introduction}\label{intro}

It has long been known that star formation preferentially occurs within molecular clouds. 
Yet \textbf{the star formation process is still not completely understood}.
It is clear that gravity is the major factor as it drives collapse motions and operates on all scales.
However, one needs additional processes that stabilise the gas \textbf{or terminate star formation quickly} in order to \textbf{explain low star formation efficiencies} observed in molecular clouds. 
Although there are many processes that act at the different scales of molecular clouds, \textbf{turbulent support has often been argued} to be the best candidate for this task.

In the literature, turbulence has an ambiguous role in the context of star formation. 
In most of the cases, turbulence is expected to stabilise molecular clouds on large scales \citep{Fleck1980,McKee1992,MacLow2003}, while feedback processes and shear motions heavily destabilise or even disrupt cloud-like structures \citep{Tan2013,Miyamoto2014}. 
However, it remains unclear whether there are particular mechanisms that dominate the driving of turbulence within molecular clouds, \textbf{as every process is supposed to be traced by typical features in the observables.
Yet, these features are either not seen or are too ambiguous to clearly reflect the dominant driving mode.}
For example, turbulence that is driven by large-scale velocity dispersions during global collapse \citep{Ballesteros2011a,Ballesteros2011b,Hartmann2012} produces P-Cygni lines \textbf{ that have not yet been observed on scales of entire molecular clouds}. 
Internal feedback, on the contrary, seems more promising as it drives turbulence outwards \citep{Dekel2013,Krumholz2014}.
Observations, though, demonstrate that the required driving sources need to act on scales of entire clouds; which typical feedback\textbf{, such as radiation, winds, jets, or supernovae (SNe),} cannot achieve \citep{Heyer2004,Brunt2009,Brunt2013}.

There have been many theoretical studies that have examined the nature and origin of turbulence within the phases of the interstellar medium \citep[ISM;][and references within]{MacLow2004}. 
The most established \textbf{characterization of turbulence in general was} by \citet{Kolmogorov1941} who investigated fully developed, incompressible turbulence \textbf{driven on scales larger than the object of interest, and diffusing on scales much smaller than those of interest}.
In the scope of this paper this object is a single molecular cloud. 
\textbf{Molecular clouds are highly compressible, though.
Only a few analytical studies have treated this case.}
\citet{She1994} and \citet{Boldyrev2002}, for example, generalise and extend the predicted scaling of the decay of turbulence to supersonic turbulence.
\citet{Galtier2011} and \citet{Banerjee2013} provide an analytic description of the scaling of mass-weighted structure functions.

In this paper, we examine three molecular clouds that formed \textbf{self-consistently} from SN-driven turbulence in the simulations by \citet[\citetalias{IbanezMejia2016} and \citetalias{IbanezMejia2017} hereafter]{IbanezMejia2016,IbanezMejia2017} and study how the turbulence within the clouds' gas evolves.
The key questions we \textbf{address} are the following: 
What dominates the turbulence within the simulated molecular clouds? 
\textbf{How can structure functions inform us about the evolutionary state of MCs and the dominant physical processes within them?}

In Sect.~\ref{methods}, we introduce the simulated clouds in the context of the underlying physics involved in the simulations.
Furthermore, we describe the theoretical basics of velocity structure functions.
Sect.~\ref{results} demonstrates that the velocity structure function is a useful tool to characterise the dominant driving mechanisms of turbulence in molecular clouds and can be applied to both simulated and observed data. 
We examine the influences of utilising one-dimensional velocity measurements, different Jeans refinement levels, density thresholds and density weighting on the applicability of the velocity structure function and the results obtained with it in Sect.~\ref{discussion}.  
\textbf{At the end of this section, we will also compare our results to observations studies.}
We summarise our findings and conclusions in Sect.~\ref{conclusions}.



\endinput
